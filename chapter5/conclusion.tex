\chapter{Conclusion} \label{chap:conclusion}

\section{Apports de la thèse} \label{sec:apports}

Dans cette thèse, j’ai cherché à apporter des éléments de compréhension et de réponse sur deux thèmes majeurs autour de la transition énergétique : le choix d’une stratégie de déploiement à long terme et l'impact sur l’emploi. 
Je résume les apports de ma recherche dans cette partie, avant de souligner les limitations de mon travail et d'ouvrir sur les pistes de recherches futures dans la dernière partie.

\subsection{Le choix d'une stratégie de déploiement des énergies renouvelables}

% Rappeler le but de la recherche
En France, le parc nucléaire atteint aujourd'hui la fin de sa durée de vie initialement prévue. Les réacteurs peuvent être rénovés et prolongés moyennant un investissement dit de «~Grand Carénage~» estimé à cent milliards d'euros par EDF. Faut-il rénover tous ces réacteurs ? Ce choix est rendu complexe par la combinaison des incertitudes et de l'inertie du système électrique.
Il existe en effet de nombreuses incertitudes sur les coûts des technologies (renouvelables et nucléaires), sur le niveau de la demande et le prix du \coo. Il n'est donc pas évident d'estimer la compétitivité relative de ces différentes options dans dix ou vingt ans. Or, il s'agit d'une donnée importante du problème, du fait de l'inertie du système électrique.
La décision de rénover ou non un réacteur doit être prise plusieurs années, voire décennies en amont, afin de préparer les travaux nécessaires ou, le cas échéant, les moyens de production alternatifs ; et le choix de rénover une centrale nucléaire ou de la remplacer par des éoliennes engage des capacités de production pour au moins vingt ans.
Comment choisir une stratégie de déploiement face à ces incertitudes et ces inerties ?

% Résumer le travail effectué
%% flore -> plusieurs optimums équivalent. Comment choisir ?
Pour répondre à cette question, j'ai commencé par construire un modèle d'optimisation du parc électrique français, baptisé FLORE. Un modèle détaillé est en effet nécessaire pour étudier finement les impacts du déploiement des énergies renouvelables sur la sécurité d'approvisionnement et sur leurs coûts d'intégration au système électrique-- ou, plus précisément, sur la valeur marginale de ces technologies à production variable pour le système.
Le modèle FLORE calcule l'optimum économique, c'est-à-dire la stratégie de moindre coût pour un ensemble de paramètres donnés. Mais beaucoup de ces paramètres sont soumis à des marges d'incertitude importantes. En faisant des simulations avec le modèle FLORE pour toutes les valeurs plausibles des paramètres, j'aboutis alors à un premier résultat : au vu des incertitudes actuelles, il n'existe pas un unique optimum. Plusieurs stratégies peuvent prétendre être la meilleure, selon l'évolution future de paramètres clés.
Ce résultat permet de mieux comprendre les nombreux débats sur la place du nucléaire en France. Il montre que, dans une certaine mesure, le choix de la meilleure stratégie relève d'un pari -- d'où le titre de l'article tiré du second chapitre : «~\textit{The French nuclear bet~»}.


%% Méthode RDM
La coexistence de plusieurs optima possibles est un résultat intéressant, mais également paralysant pour le décideur. Quelle stratégie adopter alors ? Concrètement, faut-il rénover les réacteurs les plus anciens, qui approchent de la fin de leur durée de vie initiale ? Pour le savoir, il est nécessaire de déterminer aujourd'hui une trajectoire à suivre, quitte à la modifier ensuite avec l'arrivée de nouvelles informations.
L'approche traditionnelle consistant à déterminer un unique optimum minimisant les coûts s'avère ici inopérante. En effet, le concept même d’optimum n’est plus adapté lorsque les incertitudes sont trop importantes : différentes hypothèses également plausibles mènent à différents optima, entre lesquels il est impossible de choisir. 
J'ai donc délaissé la recherche d'un optimum pour me tourner vers la recherche de stratégies robustes. Pour cela, j'ai trouvé dans le champ des sciences du climat la méthode dite de \textit{Robust Decision Making (RDM)}, et j'ai décidé de l'appliquer à l'analyse du mix électrique français. Avec cette méthode, une stratégie est définie comme robuste si elle s'avère satisfaisantes pour un grand nombre de valeurs plausibles des paramètres. Pour le dire autrement, les stratégies robustes sont celles qui offrent de bonnes performances dans un grand nombre de futurs possibles. 
Pour déterminer ces stratégies robustes, j'ai suivi les deux temps de l'approche \textit{RDM} : ouvrir puis simplifier. 
	
%%Une analyse quantitative
Dans un premier temps, j'ouvre le champ des possibles en analysant un grand nombre de scénarios. Les exercices prospectifs sont souvent construits sur un petit nombre de scénarios, généralement trois ou quatre, choisis \textit{a priori}. Cette approche a le mérite de la clarté, mais présente également le risque intrinsèque de passer à côté de certains scénarios qui pourraient s'avérer meilleurs \textit{ex post}.
A contrepieds de cette méthode usuelle, j'ai pris le parti d'étudier vingt-sept stratégies, en quadrillant au mieux l'espace des décisions possibles. Une stratégie est ici définie comme le choix de fermer certains réacteurs parmi les cinquante-huit existants. Pour chaque stratégie, afin de représenter les incertitudes, j'étudie le mix électrique correspondant pour cent huit jeux de paramètres. J'aboutis ainsi à près de 3~000 simulations.
Ces simulations permettent de retrouver les termes actuels du débat sur la place des renouvelables et du nucléaire : on y retrouve notamment les quatre positions qui ont structuré le Débat National sur la Transition Energétique en 2014. 
Je mets ainsi à jour les hypothèses qui sous-tendent chacun de ces scénarios. 
Mais mon analyse va au-delà de ce travail d'élicitation. Ell contribue à enrichir le débat en révélant de nouvelles trajectoires intéressantes, de nouvelles stratégies potentielles. 

%% une simplification
Dans un second temps, j'ai cherché à simplifier le problème, à proposer un outil d'aide à la décision. 
Pour cela, j'applique les méthodes d'analyse statistique de la méthode \textit{RDM} à mes 3~000 simulations. Je mets ainsi à jour non seulement les facteurs d'incertitudes déterminants, mais surtout les combinaisons de facteurs qui s'avèrent décisives.
Je recompose ainsi le débat en distinguant - de façon quantifiée - deux univers structurants : d'une part, un univers favorable au nucléaire avec des coûts de rénovation faibles, un progrès modéré dans les technologies renouvelables, une augmentation de la demande d'électricité, un prix élevée du gaz et/ou du CO\textsubscript{2}; d'autre part, un univers défavorable au nucléaire, qui est le complémentaire du premier.
Ces deux ensembles de futurs sont possibles, mais il est impossible de quantifier leurs probabilités relatives de façon objective. Je laisse donc ce choix, ce pari sur le futur, au décideur. 
Mais pour l'aider dans sa tâche, je détermine la meilleure stratégie robuste correspondant à chaque probabilité relative. Autrement dit, je peux proposer la meilleure stratégie robuste en ne posant qu'une seule question au décideur : «~selon vous, y a-t-il une probabilité faible, moyenne ou élevée que le futur fasse partie de l'univers décrit comme “~favorable au nucléaire”~» ?
 
% conclusion
Mes résultats indiquent que la prolongation de tous les réacteurs est loin d’être une évidence. Les incertitudes sur la demande d'électricité, sur le coût de rénovation des centrales nucléaires existantes et des énergies renouvelables, ou encore sur le prix du \coo\ plaident pour une stratégie plus diversifiée et plus flexible. 
Une rénovation de l'ensemble des réacteurs nucléaires entrainerait le risque de voir certains réacteurs fonctionner en semi-base ou en pointe, en cas de baisse de la demande d’électricité. Ce risque d'un nombre réduit d'heures de fonctionnement fait planer une ombre sur la rentabilité d'un investissement devant être décidé plusieurs décennies en amont.
En outre, une rénovation complète rend extrêmement vulnérable à un coût de rénovation des centrales plus élevé que prévu, ainsi qu'à un éventuel défaut générique. 
Ne pas prolonger tous les réacteurs et compléter les besoins de production par l'installation de petites capacités de renouvelables fournit, au contraire, une flexibilité sur le niveau d'offre.

% Conclure sur les implications
Au vu des estimations actuelles sur l'évolution des coûts, de la demande d'électricité et du prix du CO\textsubscript{2}, adopter une stratégie robuste implique de fermer 7 à 14 réacteurs. Cette proposition se démarque des deux positions les plus visibles dans le débat français, à savoir la rénovation complète et la sortie complète du nucléaire. Ces deux extrêmes sont souvent défendus sur la base d’une vision binaire comparant les coûts des technologies : la meilleure technologie est la moins chère, il faudrait donc n'installer qu'elle. Cette approche fondée uniquement sur les coûts directs fait l’impasse sur le problème des coûts d’intégration des renouvelables – ou plutôt de leur valeur marginale pour le système - et elle ignore complètement les questions d’incertitude. L'utilisation d'un modèle intégré du parc électrique était nécessaire pour prendre en compte les effets de la variabilité des renouvelables. Coupler ce modèle à la méthode de \textit{Robust Decision Making} m'a permis de prendre les incertitudes à bras-le-corps, et de déterminer de nouvelles options plus robustes pour une stratégie française de déploiement des énergies renouvelables.

\subsection{L'impact sur l'emploi}
% Rappeler le but de la recherche
Par son importance politique et sociale, l'emploi est une thématique clé de la transition énergétique. Sortir du tout-carbone nécessite des investissements massifs dans de nombreuses sphères de l'économie, notamment la production d'électricité, les transports et la rénovation des bâtiments. Ces investissements peuvent-ils être l'occasion de créer des emplois locaux et pérennes ?
Bien plus qu'un simple co-bénéfice de la lutte contre les changements climatiques, cette question semble parfois être le réel moteur de l'action publique. Le nombre d'emplois potentiels est ainsi un argument de poids à tout projet de politique énergétique. Ce constat s'applique tout particulièrement à la France, où le chômage avoisine les 10 \% depuis 2012.
Au niveau mondial, l'emploi peut aussi servir de catalyseur des négociations internationales. En effet, la réduction des émissions de gaz à effet de serre est typique d'une situation de tragédie des communs : une action collective est nécessaire mais chaque Etat peut choisir de jouer les passagers clandestins et laisser les autres faire tous les efforts. S'il existait un potentiel pour des emplois locaux, alors chaque Etat gagnerait à devenir acteur de cette transition. 
Enfin, l'importance sociale de l'emploi est soulignée de façon continue par les études sur le bien-être, ce qui renforce encore son intérêt comme objet d'étude.

Dans le débat public, un argument souvent employé est celui du contenu en emploi. Les énergies renouvelables sont souvent présentées comme ayant un fort contenu en emploi, car elles seraient plus «~locales~» que les centrales conventionnelles qui doivent importer leur combustible, et plus intensives en main-d'\oe{}uvre car moins capitalistiques. Ce fort contenu en emploi expliquerait qu'un investissement dans les énergies renouvelables génère plus d'emplois que son équivalent dans les énergies fossiles ou fissiles. Mais le concept même de contenu en emploi reste flou. Que recouvre ce terme exactement, et comment est-il relié aux notions de production locale et intensive en travail ? Comment expliquer les variations de contenu en emploi entre les différents secteurs productifs ?

Dans la littérature académique, de nombreuses études quantifient l'impact net sur l'emploi de politiques de transition énergétique. Cependant, les conclusions de ces simulations varient, non seulement quant au nombre mais aussi quant au signe, positif ou négatif. Cette diversité des résultats, combinée à la variété des scénarios étudiés et des choix de modélisation (type de modèle, valeurs des paramètres, etc.) fait qu'il est difficile d'établir des conclusions fermes. 
Quels résultats peuvent être imputés à des choix discutables de modélisation et quelles conclusions semblent au contraire invariantes et robustes ? C'est en passant par une compréhension fine des mécanismes sous-jacents aux modèles que je tente de répondre à ces questions.

Je me suis penché sur la question du contenu en emploi dans le chapitre \ref{chap:TE_Emploi}. Je commence par proposer une métrique pour ce terme. En effet, plusieurs définitions apparaissent dans la littérature, mais conduisent à des biais pour estimer les emplois nets. Je retiens la définition d’un nombre d’ETP par unité monétaire de demande finale.
Ensuite, j’ai mesuré ce contenu en emploi pour différentes branches de l’économie. En croisant ce contenu en emploi avec le contenu en émissions, je mets ainsi en évidence quelques secteurs faiblement émetteurs et fortement créateurs d’emplois. Réorienter l’investissement vers ces secteurs satisfait donc au double impératif d’une transition écologique et sociale – sous réserve de rester dans les limites de validité du modèle input-output utilisé.

Mais la vraie originalité de ce chapitre a été de décomposer le contenu en emploi des différentes branches, pour faire apparaître quatre facteurs : le taux d’importations finales, le taux d’importations intermédiaires, le niveau de taxes et de subventions, la part des salaires dans la valeur ajoutée, et enfin le niveau des salaires.
Cette décomposition permet d’améliorer la compréhension de ce qu’est le contenu en emploi. En l’appliquant, il est possible de déterminer si le contenu en emploi d’une branche est élevé grâce à ses faibles importations, une part importante du travail dans la valeur ajoutée, de faibles taxes, ou à cause de bas salaires.
Cette nouvelle information a notamment révélé un biais du contenu en emploi, relatif à l’impact des taxes et subventions. Les secteurs fortement subventionnés -- notamment celui de l'agriculture en France -- voient leur contenu en emploi augmenter \textit{ceteris paribus}. Mais ces subventions devront être financées par des prélèvements ailleurs dans l’économie, avec un impact négatif qui n’est pas pris en compte dans le contenu en emploi. Cette décomposition permet de mesurer ce biais, et donc de le corriger. 
En outre, notre méthodologie de décomposition peut être appliquée de façon prospective, pour analyser les créations d’emplois liées à la réallocation de la demande finale d’un secteur vers un autre (avec toutes les précautions nécessaires liées à l’utilisation des modèles IO). Ces informations peuvent être utiles à l’économiste et au décideur : l’impact économique, social et politique est en effet différent si les nouveaux emplois proviennent d’une réduction des importations ou de salaires plus faibles. Dans un cas, il s’agit de relocaliser la production au détriment d’autres pays, donc d’une redistribution internationale des richesses ; dans l’autre, de partager les revenus du travail, avec des conséquences redistributives nationales.
Cette décomposition permet de mieux comprendre les mécanismes à l’œuvre dans les modèles macro-économiques. Encourager un secteur intensif en main-d’œuvre plutôt qu’en capital va augmenter la demande de travail ; réduire les importations va augmenter la valeur ajoutée locale ; baisser les salaires peut avoir un impact négatif sur le pouvoir d’achat. L’idée n’est bien sûr pas de se substituer aux modèles macro-économiques, mais bien d’apporter un éclairage sur les mécanismes économiques qu’ils mettent en jeu.

Dans le chapitre \ref{chap:mechanisms}, je poursuis la réflexion sur l'impact d'une réallocation de la demande finale, mais je m'attache à mettre à jour les mécanismes économiques sous-jacents. Je compare deux types de modèles : des modèles CGE et des modèles Input-Output, afin de faire ressortir leurs différences, mais aussi d'identifier des résultats robustes pour ces deux modèles. Mes recherches portent ainsi une réflexion sur l’usage et la validité des modèles économiques utilisés pour les études en emploi liées à la transition énergétique. 

Dans la continuité du chapitre précédent, je mets en évidence et j’étudie trois phénomènes à l'aide de modèles stylisés : la part des rémunérations du travail dans la valeur ajoutée, le niveau des salaires et le taux d'importations.
Je montre que les différences de résultat entre les modèles dépendent de l'origine du mécanisme économique présidant à la création d'emploi : 
\begin{itemize}
	\item Investir un secteur avec une forte part de travail crée de l'emploi dans les deux modèles, mais légèrement moins dans le CGE. Cependant, cet alignement apparent est en fait la combinaison de deux effets contraires. Dans le modèle CGE, cet investissement génère une moindre augmentation de la demande de travail qu'en IO, car il a un effet rétroactif sur les salaires qui les pousse à la hausse, ainsi qu'une substitution du capital au travail. Ces effets n'apparaissent pas en IO. Mais dans le CGE, on observe également une baisse des prix du capital qui augmente le pouvoir d'achat et donc l'offre de travail. En IO, les prix sont fixes et donc cet effet n'apparait pas.
	\item Encourager un secteur avec de faibles salaires conduit au même effet positif dans les deux types de modèles. Ce résultat est confirmé pour différentes modélisations du marché du travail, que l'origine de ces faibles salaires s'explique par des rigidités dans la mobilité des travailleurs ou par un fort taux de travailleurs peu qualifiés.
	\item Enfin, cibler des secteurs peu importateurs conduit à des effets très positifs en IO, avec une amélioration nette de la balance commerciale, du PIB et de l'emploi. A l'inverse, dans un modèle CGE, cet effet commercial n'apparait pas : cibler des secteurs peu importateurs n'entraine aucun effet positif. Les exportations et la consommation privée diminuent, rétablissant ainsi l'équilibre de la balance commerciale.
\end{itemize}

Après ce travail d'analyse qualitative, je me tourne vers l'étude quantitative et numérique de deux politiques : le déploiement de panneaux solaires et l'isolation thermique de bâtiments. 
En utilisant un modèle CGE standard à 58 secteurs calibré sur les données françaises, je montre que ces deux politiques permettent de générer de l'emploi, du fait de la grande part de travail dans la valeur ajoutée et des salaires faibles. Les résultats des deux modèles sont proches pour le solaire, et légèrement supérieurs en IO pour l'isolation.

Enfin, je montre que de nombreux modèles CGE sous-estiment l’impact sur l’emploi à cause d'une mauvaise calibration de leur fonction de production. Les fonctions de type Cobb-Douglas sont en effet largement utilisées, car très pratiques d’un point de vue calculatoire. Mais leur utilisation équivaut à l'hypothèse implicite d’une élasticité capital-travail unitaire. Or, des études empiriques montrent que cette élasticité est significativement inférieure à 1, et varie plutôt de 0,2 à 0,6 \citep{VanderWerf2008}. Dans mes scénarios, cette mauvaise valeur de l’élasticité réduit les effets positifs sur l’emploi de 20 à 40 \% par rapport aux estimations basées sur des valeurs d'élasticité empiriques. 

Au final, mes résultats indiquent qu’il est possible de créer des emplois en investissant dans des secteurs intensifs en travail plutôt qu’en capital, et où les salaires sont plus faibles. L'économie et l'écologie peuvent donc aller de pair. Il s’agit d’un message fort pour accélérer la transition énergétique.
Mais mon analyse appelle également à utiliser avec prudence les annonces quantifiées en termes d’emplois : les modèles IO auront tendance à être optimistes, et de nombreux modèles CGE seront au contraire pessimistes.
Enfin, mon travail remet en cause l’argument souvent utilisé du bénéfice à développer des énergies renouvelables locales, qui permettraient de réduire les importations et ainsi créer des emplois. Le poids politique de cet argument et son côté intuitif semblent inversement proportionnels à sa robustesse dans les analyses économiques.


\section{Ouverture}

Je vois bien sûr de nombreuses pistes pour améliorer et étendre les réflexions menées dans cette thèse, tant sur la recherche prospective de stratégies que sur la thématique de l'emploi.

\paragraph{Articuler prospective de long terme et innovations technologiques} \hfill

Concernant le déploiement des énergies renouvelables, mon modèle d'optimisation du parc électrique pourrait bénéficier de plusieurs extensions. Afin de pouvoir faire mes nombreuses simulations, j’ai dû réaliser des arbitrages entre temps de calcul et précision des représentations. Mais certains aspects du modèle gagneraient à être plus détaillés.

Un premier axe concerne la représentation des contraintes de flexibilité du système. Je modélise de façon très simple la variabilité des énergies renouvelables variables et je n'intègre pas explicitement les batteries, qui auront peut-être un rôle clé à jouer dans les décennies à venir. Mais il est difficile d'anticiper ces évolutions technologiques, et l'inclusion de batteries mène à une explosion combinatoire. 
Pour la variabilité de la production renouvelable, les erreurs de prédiction ont fortement diminué depuis quelques années. En Espagne, l’erreur de prévision une heure en amont a été divisée par trois en cinq ans, passant de 12~\% à 4~\%, avant de se stabiliser \citep{IRENA2017}. La réglementation sur le niveau minimal de réserves pourrait également évoluer, avec potentiellement une baisse importante des coûts d'intégration des énergies renouvelables \citep{Lorenz2017}. Il serait intéressant de regarder plus avant ces questions, afin de s’assurer que mes résultats ne sont pas affectés. 

Le volet sur la demande d’électricité pourrait également être affiné. L’importance des politiques de demande en complément de la politique d’offre a été soulignée à de nombreuses reprises par l'IDDRI \citep{Berghmans2017, Rudinger2017}. Une modélisation plus avancée des secteurs économiques, et de leur capacité à réduire ou déplacer leur consommation, apporterait un réalisme et une précision accrus. L'enjeu est triple : modéliser le niveau de la pointe, le volume annuel et la flexibilité de la demande. Ce travail pourrait s'inspirer des analyses réalisées par \citet{RTE2016} dans son \textit{Bilan Prévisionnel 2016} ou encore de celles de l'\citet{ADEME2015} sur la faisabilité d'un mix 100 \% renouvelable.

Mais le plus intéressant est peut-être à chercher du côté des nouvelles technologies émergentes. 
Les éoliennes offshores semblent être à l’aube d’une expansion rapide. Fin 2016, la capacité cumulée atteignait 12,6 GW. Et cette capacité pourrait doubler d’ici 2020. Ce boom est accompagné d’une chute rapide des prix \citep[p.14]{IRENA2017a}. Un rapport de McKinsey publié en mai 2017 s’intitule d’ailleurs «~Winds of change? Why offshore wind might be the next big thing~» \citep{McKinsey2017Wind}. En novembre 2016, un appel d’offre au Danemark s’est conclu au prix de 49,9 euros par mégawattheure. Un record absolu pour cette technologie, qui se rapproche rapidement de la compétitivité avec les autres sources d’électricité (pour comparaison, le tarif d'achat du nucléaire historique en France est de 42 euros/MWh, et celui des éoliennes de 82 euros/MWh). Les éoliennes offshores sont intéressantes à double titre : leur installation rencontre moins d’oppositions locales que les éoliennes terrestres, et leur production est plus stable, ce qui augmente la valeur de leur production pour le système électrique \citep{Hirth2016}.

Le power-to-gas est également une source d’énergie qui pourrait s’avérer importante pour l’avenir. Elle permet de bénéficier du faible coût variable des renouvelables, et peut participer à gérer les surplus de production. Son intérêt s’étend même au-delà du seul secteur électrique, avec la possibilité de fabriquer du gaz ensuite réinjecté sur le réseau, et utilisé directement par les secteurs industriels ou tertiaires. Cette technologie crée donc un pont entre les secteurs électriques et gaziers, dont il serait intéressant d’explorer les conséquences. En janvier 2016, plus de 50 projets pilotes étaient lancés à travers le monde \citep{EneaConsulting2016}. Et cette technique est l’une des clés de voûte du scénario négaWatt, ce qui souligne le rôle important que peut avoir cette technologie pour une politique ambitieuse de réduction des émissions.

Le potentiel de ces nouvelles technologies pose d'ailleurs une question plus fondamentale encore : comment faire de la prospective de long terme dans un environnement technologique si mouvant ? Cette question mériterait sans doute une thèse à elle seule. Le développement très rapide des énergies renouvelables a ainsi été largement sous-estimé par de nombreux experts, y compris par l'institution de référence qu'est l'Agence Internationale de l'Energie \citep{Metayer2015, Creutzig2017}. A l'inverse, le potentiel des technologies de captage et stockage du carbone a suscité un temps de fortes espérances, avant d'être ensuite drastiquement revu à la baisse. Ces leçons du passé soulignent à quel point il est possible de se tromper lourdement à moyen et même court terme. Une forme d'humilité m'apparait donc nécessaire en prospective. L'ampleur des incertitudes ne peut qu'être sous-estimée. Une bonne question me parait alors : quelles conclusions peut-on extraire des travaux de prospective malgré ces incertitudes ? 
En l'état, ma recherche sur le mix électrique fournit un outil et une méthode déterminer les meilleures stratégies au vu des incertitudes actuelles, avec les technologies actuellement matures. Mes analyses et résultats pourraient être mis à jour régulièrement afin d’intégrer les nouvelles informations en termes de coûts, de demande d’électricité ou de prix du CO\textsubscript{2}. Cette approche ne permet certes pas d'anticiper les ruptures technologiques fortes; mais je ne vois pas aujourd'hui de meilleure méthode.

\paragraph{Intégrer l'emploi dans la décision publique} \hfill

Une autre question parcourt en filigrane ma thèse : comment intégrer les impacts sur l'emploi dans la décision publique ?
Aujourd'hui, de nombreuses politiques font l'objet d'une analyse en termes de coût et d'une analyse des impacts sur l'emploi. Mais ces deux mesures sont juxtaposées, sans lien direct. La façon dont l'emploi pèse dans la décision n'est alors pas claire.

Coût économique et impact sur l'emploi ne vont pas de pair : ils peuvent différer quant à l'ampleur relative de leurs résultats, mais aussi sur leur signe, positif ou négatif \citep{Patuelli2005}. 
En particulier, je montre dans les chapitres \ref{chap:TE_Emploi} et \ref{chap:mechanisms} que, à coût égal, des technologies avec une plus grande part de rémunérations salariales dans la valeur ajoutée ou des salaires plus faibles génèrent davantage d'emplois. 

Comment faire du calcul économique intégrant ces deux aspects, coût direct et emplois ? Comment articuler les analyses coût-bénéfice sectorielles avec les retombées macro-économiques ?
Ces questions, Malinvaud lui-même se les posait dans un rapport sur la résorption des déséquilibres économiques \citep{CommissariatgeneralduPlan1984}. Plus récemment, \citet{Masur2011, Masur2012} poursuivent cette même réflexion.
Est-il possible d'intégrer directement le coût du chômage à l'analyse coût-bénéfice, en lui assignant une valeur monétaire ? Et plus généralement, quelle valeur attribuer à l'emploi ? 

%Cette question sociétale, l'économiste doit se la poser en lien avec les autres sciences sociales.
Ces questions constituent un enjeu sociétal, puisque les réponses peuvent influencer les décisions publiques. Et la recherche sur ce sujet ne pourra se faire qu'en lien avec les autres branches de l'économie, notamment l'économie comportementale, l'économie du bien-être et l'économie écologique, y compris dans ses branches non-monétaires.
%\citet{Fourgeaud1986} a cherché une solution par les prix duaux (\textit{shadow prices}).

\paragraph{Réconcilier la macroéconomie avec les sciences sociales et empiriques} \hfill

Le dernier axe de recherche que je souhaite souligner concerne les hypothèses fondamentales de la macroéconomie.
Dans cette thèse, mes questionnements ont porté sur la pertinence et l’importance des choix de modélisations sur les résultats. Cela s’est traduit, dans le chapitre \ref{chap:mechanisms}, par la comparaison de modèles, par l’analyse de nombreuses spécifications différentes et de multiples tests de sensibilité : sur l’ajustement de la balance commerciale, le marché de travail, etc. 

Ces travaux pourraient être complétés par une analyse des effets transitoires et du rôle de la monnaie. Je pense en particulier aux modèles de type nouveau-keynésien, tel celui utilisé par \citet{Blanchard2007} pour observer l'impact d'un choc pétrolier.\footnote{«~The Macroeconomic Effects of Oil Price Shocks: Why are the 2000s so different from the 1970s?~»}
Les modèles DGSE sont très utilisés dans la communauté des modélisateurs macroéconomiques, mais leur absence de réalisme empirique m'interpelle. \citet{Stiglitz2008} émet d'ailleurs une critique virulente de ces modèles, allant jusqu'à affirmer : 
\begin{quote}
	\textit{
			«\,I believe that most of the core constituents of the DSGE model are flawed -- sufficiently badly flawed that they do not provide even a good starting point for constructing a good macroeconomic model\,».
	}
\end{quote}
Sans vouloir entrer ici dans une longue discussion sur les qualités et les défauts des modèles DSGE, ces critiques motivent aujourd'hui ma préférence pour des modèles plus simples et à plusieurs secteurs comme les modèles nouveau-keynésiens.

Mais au-delà du type de modèle, mes interrogations s'étendent aujourd'hui à certaines hypothèses plus fondamentales encore. 
Le questionnement de l’utilisation d’un agent représentatif, parfaitement rationnel et informé, me paraît un axe de recherche attractif. Bien sûr, les critiques de cet \textit{homo \oe{}conomicus} ne sont pas nouvelles. Carl Menger critiquait l’hypothèse implicite d’homogénéité des individus ; Keynes l’hypothèse d’information complète ; Veblen la rationalité parfaite ; \citet{Bourdieu1977} l’a qualifié de «~monstre anthropologique~», et \citet{Sen2012} de «~demeuré social~» ; plus récemment, les travaux de l’économie comportementale ont mis en évidence de nombreux biais cognitifs \citep{Thaler2009,Kahneman2011}.
Mais l’ampleur des critiques souligne le chemin qui reste à parcourir. Des pistes qui me semblent intéressantes ont trait à la compréhension et la modélisation du comportement des agents. L'hypothèse de «~rationalité parfaite~», malgré son appellation flatteuse, est finalement une représentation très stylisée de la façon dont une personne prend ses décisions. Il devrait donc être possible d'enrichir cette représentation, de se rapprocher des comportements observés dans les autres sciences sociales. Par exemple, des travaux récents explorent l’impact d’agents «~presque parfaitement rationnels~» (near-rational expectations). Ainsi, \citet{Farhi2016} montre que l’effet des politiques monétaires est alors plus faible que ce qu’indiquent les modèles à agents rationnels.

Par ailleurs, l’explosion des capacités computationnelles ouvre le champ à l’introduction de modèles avec de nombreux agents représentatifs – plusieurs centaines ou milliers –, ce qui pourrait permettre de dégager des dynamiques agrégées qui échappent à des modèles représentant seulement un ou quelques agents représentatifs. Des travaux récents vont dans cette direction. Ainsi, \citet{Kaplan2016} montrent que les effets de la politique monétaires sont très différents dans des modèles à agents représentatifs et dans des modèles à agents hétérogènes.

Représenter explicitement des agents multiples, dans leur diversité et leur imperfection, me paraît être une piste de recherche intéressante à double titre. Elle peut permettre à l’économie de renforcer ses fondements empiriques, de poser ainsi des hypothèses vérifiables et réfutables – critère de scientificité selon Popper – et d’abandonner, le cas échéant, des hypothèses contredites par l’expérience. Englober la théorie du choix rationnel dans un modèle plus global de l’action peut être l’occasion de renouer des liens avec les autres sciences sociales, notamment l’anthropologie, la psychologie et la sociologie. «~L'économie qui est la science sociale mathématiquement la plus avancée, est la science socialement la plus arriérée, car elle s'est abstraite des conditions sociales, historiques, politiques, psychologique, écologiques inséparables des activités économiques~» constate \citet{Morin1999}. Sortir de ce fonctionnement en silos, se rapprocher des autres disciplines pour mieux appréhender la complexité de l'humain : voilà qui me paraît une piste prometteuse de recherche future.





