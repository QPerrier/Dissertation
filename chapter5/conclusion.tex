\chapter{Conclusion} \label{chap:conclusion}

\section{Apports de la thèse}

Dans cette thèse, j’ai cherché à apporter des éléments de compréhension et de réponse sur deux thèmes majeurs autour des énergies renouvelables : le choix d’une stratégie de déploiement à long terme et l'impact sur l’emploi. 

\subsection{Le choix d'une stratégie de déploiement des énergies renouvelables}

La réflexion sur le choix d’une stratégie est menée dans le chapitre \ref{chap:nuclear_bet}, où j'analyse le cas du secteur électrique français. L’originalité de ma recherche consiste à intégrer pleinement les questions d’incertitudes dans la réflexion sur une trajectoire de mix électrique.
Cette nouvelle approche conduit à un renversement significatif par rapport aux méthodes traditionnelles consistant à déterminer un unique optimum minimisant les coûts. En effet, le concept même d’optimum n’est plus adapté lorsque les incertitudes sont trop importantes : différentes hypothèses également plausibles mènent à différents optima, entre lesquels il est impossible de choisir. Je me suis donc tourné vers la recherche de stratégies robustes, c’est-à-dire satisfaisantes pour un grand nombre de valeurs plausibles des paramètres. 
Dans cette optique, j'ai construit un modèle d'optimisation du parc électrique français, FLORE, afin de pouvoir étudier les impacts des énergies renouvelables sur les coûts du système électrique. Je me suis également appuyé la méthode dite de \textit{Robust Decision Making (RDM)}, développée et utilisée dans le champ des sciences du climat, que j'ai appliqué de façon originale à l'analyse d'un mix électrique. 
Avec cette approche, je n’apporte pas une unique solution «~rationnelle~», mais je montre au contraire que différentes options sont défendables, selon l’hypothèse portée sur l’évolution de paramètres futurs. 

%Une analyse quantitative
Dans un premier temps, j'ouvre le champ des possibles en analysant un grand nombre de scénarios. Les exercices prospectifs sont souvent construits sur un petit nombre de scénarios, généralement trois ou quatre, choisis \textit{a priori}. Cette approche a le mérite de la clarté, mais présente également le risque intrinsèque de passer à côté de certains scénarios qui pourraient s'avérer meilleurs \textit{ex post}.
A contrepieds de ce \textit{modus operandi} usuel, j'ai pris le parti d'étudier vingt-sept stratégies possibles, en quadrillant au mieux l'espace des décisions possibles. Ces vingt-sept stratégies permettent de retrouver les termes actuels du débat sur la place des renouvelables et du nucléaire : on y retrouve notamment les quatre positions qui ont structuré le Débat National sur la Transition Energétique en 2014. Mais mes scénarios vont plus loin et contribuent à enrichir ce débat en révélant de nouvelles trajectoires intéressantes, de nouvelles stratégies potentielles. 

% une simplification
Dans un second temps, j'ai cherché à proposer un outil d'aide à la décision en appliquant la méthode \textit{RDM} aux vingt-sept scénarios. Je mets à jour non seulement les facteurs d'incertitudes déterminants, mais surtout les combinaisons de facteurs qui s'avèrent décisives.
Je recentre ainsi le débat en distinguant - de façon quantifiée, grâce à des analyses statistiques - deux univers structurants : un univers favorable au nucléaire avec des coûts de rénovation faible, un faible progrès dans les renouvelables, une augmentation de la demande d'électricité, un prix élevée du gaz et/ou du CO\textsubscript{2}; et un univers défavorable au nucléaire, qui est le complémentaire du premier.
Je trace ensuite la frontière d'efficacité entre les différentes stratégies, permettant à un décideur de choisir la meilleure stratégie en fonction de l'idée qu'il se fait de la probabilité relative de chaque univers. 
Au final, ce travail permet de faciliter le pari que doivent faire les dirigeants devant prendre une décision en situation d’incertitude. ll permet de proposer la meilleure stratégie en ne posant qu'une seule question au décideur : «~selon vous, y a-t-il une probabilité forte, moyenne ou faible d'être que le futur correspond à l'univers décrit comme “~favorable au nucléaire”~» ?
 
% conclusion
Mon analyse montre que la prolongation de tous les réacteurs est loin d’être une évidence. La baisse des coûts du renouvelables, l’incertitude sur la demande, le prix du CO\textsubscript{2} et le coût de rénovation des centrales nucléaires existantes plaident pour une stratégie plus diversifiée et plus flexible. 
Une rénovation complète entrainerait le risque de voir certains réacteurs nucléaires fonctionner en semi-base ou pointe, en cas de baisse de la demande d’électricité. Ce risque d'un nombre réduit d'heures de fonctionnement fait planer une ombre sur la rentabilité d'un investissement devant être décidé plusieurs décennies en amont.
Une rénovation complète rend en outre extrêmement vulnérable à un coût de rénovation des centrales plus élevé que prévu, ainsi qu'à un éventuel défaut générique. 
Ne pas prolonger tous les réacteurs et compléter la production par l'installation de petites capacités de renouvelables offre, au contraire, une flexibilité sur le niveau d'offre.

Au vu des estimations actuelles sur l'évolution des coûts, de la demande d'électricité et du prix du CO\textsubscript{2}, adopter une stratégie robuste implique de fermer 7 à 14 réacteurs. Cette proposition se démarque des deux positions les plus visibles dans le débat français, à savoir la rénovation complète et la sortie complète du nucléaire. Ces deux extrêmes sont souvent défendus sur la base d’une vision binaire comparant les coûts des technologies : la meilleure technologie est la moins chère, et il faudrait n'installer que celle-là. Cette approche fondée uniquement sur les coûts directs fait ainsi l’impasse sur le problème des coûts d’intégration des renouvelables – ou, pour le dire autrement, de leur valeur marginale pour le système - et elle ignore complètement les questions d’incertitudes. L'utilisation d'un modèle intégré du parc électrique était nécessaire pour prendre en compte les coûts d'intégration et la variabilité des renouvelables. Coupler ce modèle à la méthode de \textit{Robust Decision Making} m'a permis de prendre les incertitudes à bras-le-corps, et de déterminer de nouvelles options plus robustes pour une stratégie française de déploiement des énergies renouvelables.

\subsection{L'impact sur l'emploi}

Les chapitres \ref{chap:TE_Emploi} et \ref{chap:mechanisms} sont centrés sur le deuxième aspect de ma thèse : la question de l’emploi dans la transition énergétique.

Le chapitre \ref{chap:TE_Emploi} a permis de clarifier la notion de contenu en emploi.  Je commence par proposer une métrique pour ce terme. En effet, plusieurs définitions apparaissent dans la littérature, mais conduisent à des biais pour estimer les emplois nets. Je retiens la définition d’un nombre d’ETP par unité monétaire de demande finale.
Ensuite, j’ai mesuré ce contenu en emploi pour différentes branches de l’économie. En croisant ce contenu en emploi avec le contenu en émissions, je mets ainsi en évidence quelques secteurs faiblement émetteurs et fortement créateurs d’emplois. Réorienter l’investissement vers ces secteurs satisfait donc au double impératif d’une transition écologique et sociale – sous réserve de rester dans les limites de validité du modèle input-output utilisé.

Mais la vraie originalité de ce chapitre a été de décomposer le contenu en emploi des différentes branches, pour faire apparaître quatre facteurs : le taux d’importations finales, le taux d’importations intermédiaires, le niveau de taxes et de subventions, la part des salaires dans la valeur ajoutée, et enfin le niveau des salaires.
Cette décomposition permet d’améliorer la compréhension de ce qu’est le contenu en emploi. En l’appliquant, il est possible de déterminer si le contenu en emploi d’une branche est élevé grâce à ses faibles importations, une part importante du travail dans la valeur ajoutée, de faibles taxes, ou à cause de bas salaires.
Cette nouvelle information a notamment révélé un biais du contenu en emploi, relatif à l’impact des taxes et subventions. Les secteurs fortement subventionnés -- notamment celui de l'agriculture en France -- voient leur contenu en emploi augmenter \textit{ceteris paribus}. Mais ces subventions devront être financées par des prélèvements ailleurs dans l’économie, avec un impact négatif qui n’est pas pris en compte dans le contenu en emploi. Notre décomposition permet de mesurer ce biais, et donc de le corriger. 
En outre, notre méthodologie de décomposition peut être appliquée de façon prospective, pour décomposer les créations d’emplois liées à la réallocation de la demande finale d’un secteur vers un autre (avec toutes les précautions nécessaires liées à l’utilisation des modèles IO). Ces informations peuvent être utile à l’économiste et au décideur : l’impact économique, social et politique est en effet différent si les nouveaux emplois proviennent d’une réduction des importations ou de salaires plus faibles. Dans un cas, il s’agit de relocaliser la production, au détriment d’autres pays, donc d’une redistribution internationale des richesses ; dans l’autre, de partager les revenus du travail, avec des conséquences redistributives nationales.
Cette décomposition permet de mieux comprendre les mécanismes à l’œuvre dans les modèles macro-économiques. Encourager un secteur intensif en main-d’œuvre plutôt qu’en capital va augmenter la demande de travail ; réduire les importations va augmenter la valeur ajoutée locale ; baisser les salaires peut avoir un impact négatif sur le pouvoir d’achat. L’idée n’est bien sûr pas de se substituer aux modèles macro-économiques, mais bien d’apporter un éclairage sur les mécanismes économiques qu’ils mettent en jeu.


Dans le chapitre \ref{chap:mechanisms}, je poursuis la réflexion sur l'impact d'une réallocation de la demande finale, mais je m'attache à mettre à jour les mécanismes économiques sous-jacents. Je compare deux types de modèles : des modèles CGE et des modèles Input-Output, afin de faire ressortir leurs différences, mais aussi d'identifier des résultats robustes pour ces deux modèles. Mes recherches portent ainsi une réflexion sur l’usage et la validité des modèles économiques utilisés pour les études en emploi liées à la transition énergétique. 

Dans la continuité du chapitre précédent, je mets en évidence et j’étudie trois phénomènes à l'aide de modèles stylisés : la part des rémunérations du travail dans la valeur ajoutée, le niveau des salaires et le taux d'importations.
Je montre que les différences de résultat entre les modèles dépendent de l'origine du mécanismes économiques présidant à la création d'emploi : 
\begin{itemize}
	\item Investir un secteur avec une forte part de travail crée de l'emploi dans les deux modèles, mais légèrement plus dans le CGE. Cependant, cet alignement apparent est en fait la combinaison de deux effets contraires. Dans le modèle CGE, cet investissement génère une moindre augmentation de la demande de travail qu'en IO, car il a un effet rétroactif sur les salaires qui les pousse à la hausse, ainsi qu'une substitution du capital au travail. Ces effets n'apparaissent pas en IO. Mais dans le CGE, on observe également une baisse des prix du capital qui augmente le pouvoir d'achat et donc l'offre de travail. En IO, les prix sont fixes et donc cet effet n'apparait pas.
	\item Encourager un secteur avec de faibles salaires conduit au même effet positif dans les deux types de modèles. Ce résultat est confirmé pour différentes modélisations du marché du travail, que l'origine de ces faibles salaires s'explique par des rigidités dans la mobilité des travailleurs ou par un fort taux de travailleurs peu qualifiés.
	\item Enfin, cibler des secteurs peu importateurs conduit à des effets très positifs en IO, avec une amélioration nette de la balance commerciale, du PIB et de l'emploi. A l'inverse, dans un modèle CGE, cet effet commercial n'apparait pas : cibler des secteurs peu importateurs n'entraine aucun effet positif. Les exportations et la consommation privée diminuent, rétablissant ainsi l'équilibre de la balance commerciale.
\end{itemize}

Après ce travail d'analyse qualitative, je me tourne vers l'étude quantitative et numérique de deux politiques : le déploiement de panneaux solaires et l'isolation thermique de bâtiments. 
En utilisant un modèle CGE standard à 58 secteurs calibré sur les données françaises, je montre que ces deux politiques permettent de générer de l'emploi, du fait de la grande part de travail dans la valeur ajoutée et des salaires faibles. Les résultats des deux modèles sont proches pour le solaire, et légèrement supérieurs en IO pour l'isolation.

Enfin, je montre que de nombreux modèles CGE sous-estiment l’impact sur l’emploi à cause d'une mauvaise calibration de leur fonction de production. Les fonctions de type Cobb-Douglas sont en effet largement utilisées, car très pratiques d’un point de vue calculatoire. Mais leur utilisation équivaut à l'hypothèse implicite d’une élasticité capital-travail unitaire. Or, des études empiriques montrent que cette élasticité est significativement inférieure à 1, et varie plutôt de 0,2 à 0,6 \citep{VanderWerf2008}. Dans mes scénarios, cette mauvaise valeur de l’élasticité réduit les effets positifs sur l’emploi de 20 à 40 \% par rapport aux estimations basées sur des valeurs d'élasticité empiriques. 

Au final, mes résultats indiquent qu’il est possible de créer des emplois en investissant dans des secteurs intensifs en travail plutôt qu’en capital, et où les salaires sont plus faibles. L'économie et l'écologie peuvent donc aller de pair. Il s’agit d’un message fort pour accélérer la transition énergétique.
Mais mon analyse appelle également à utiliser avec prudence les annonces quantifiées en termes d’emplois : les modèles IO auront tendance à être optimistes, et de nombreux modèles CGE seront au contraire pessimistes.
Enfin, mon travail remet en cause l’argument souvent utilisé du bénéfice à développer des énergies renouvelables locales, qui permettraient de réduire les importations et ainsi créer des emplois. Le poids politique de cet argument et son côté intuitif semblent inversement proportionnels à sa robustesse dans les analyses économiques.


\section{Ouverture}

Je vois bien sûr de nombreuses pistes pour améliorer et étendre les réflexions menées dans cette thèse. 

Concernant le déploiement des énergies renouvelables, mon modèle d'optimisation du parc électrique pourrait bénéficier de plusieurs améliorations. Pour des raisons de temps de calcul, j’ai en effet dû réaliser des arbitrages de modélisation. 

Une première faiblesse du modèle a trait à la représentation des contraintes de flexibilité du système. Je modélise de façon très simple la variabilité des énergies renouvelables variables et je n'intègre pas explicitement les batteries, qui auront peut-être un rôle clé à jouer dans les décennies à venir. Mais il est difficile d'anticiper ces évolutions technologiques, et l'inclusion de batteries mène à une explosion combinatoire. 
Pour la variabilité de la production renouvelable, les erreurs de prédiction ont fortement diminué depuis quelques années. En Espagne, l’erreur de prévision une heure en amont a été divisée par trois en cinq ans, passant de 12~\% à 4~\%, avant de se stabiliser \citep{IRENA2017}. La réglementation sur le niveau minimal de réserves pourrait également évoluer, avec potentiellement une baisse importante des coûts d'intégration des énergies renouvelables \citep{Lorenz2017}. Il serait intéressant de regarder plus avant ces questions, afin de s’assurer que mes résultats ne sont pas affectés. 

Le volet sur la demande d’électricité pourrait également être affiné. L’importance des politiques de demande en complément de la politique d’offre a été soulignée à de nombreuses reprises par l’IDDRI \citep{Berghmans2017}. Une modélisation plus avancée des secteurs, et de leur capacité à réduire et déplacer leur consommation actuelle ou à venir, apporterait un réalisme et une précision accrus. L'enjeu est triple : modéliser le niveau de la pointe, le volume annuel et la flexibilité de la demande. Ce travail pourrait s'inspirer des analyses réalisées par \citet{RTE2016} dans son \textit{Bilan Prévisionnel 2016} ou encore de celles de l'\citet{ADEME2015} sur la faisabilité d'un mix 100 \% renouvelable.

Mais le plus intéressant est peut-être à chercher du côté des nouvelles technologies émergentes. 
Les éoliennes offshores semblent être à l’aube d’une expansion rapide. Fin 2016, la capacité cumulée atteignait 12,6 GW. Et cette capacité pourrait doubler d’ici 2020. Ce boom est accompagné d’une chute rapide des prix. Un rapport de McKinsey publié en mai 2017 s’intitule d’ailleurs «~Winds of change? Why offshore wind might be the next big thing~» \citep{McKinsey2017Wind}. En novembre 2016, un appel d’offre au Danemark s’est conclu au prix de 49.9 euros par MWh. Un record absolu pour cette technologie, qui se rapproche rapidement de la compétitivité avec les autres sources d’électricité (pour comparaison, le tarif d'achat du nucléaire historique en France est de 42 euros/MWh, et celui des éoliennes de 82 euros/MWh). Les éoliennes offshores sont intéressantes à double titre : leur installation rencontre moins d’oppositions locales que les éoliennes terrestres, et leur production est plus stable, ce qui augmente la valeur de leur production pour le système électrique \citep{Hirth2016}.

Le power-to-gas est également une source d’énergie qui pourrait s’avérer importante pour l’avenir. Elle permet de bénéficier du faible coût variable des renouvelables, et peut participer à gérer les surplus de production. Son intérêt s’étend même au-delà du seul secteur électrique, avec la possibilité de fabriquer du gaz ensuite réinjecté sur le réseau, et utilisé directement par les secteurs industriels ou tertiaires. Cette technologie crée donc un pont entre les secteurs électriques et gaziers, dont il serait intéressant d’explorer les conséquences. En janvier 2016, plus de 50 projets pilotes étaient lancés à travers le monde \citep{EneaConsulting2016}. Et cette technique est l’une des clés de voûte du scénario négaWatt, ce qui souligne le rôle important que peut avoir cette technologie pour une politique ambitieuse de réduction des émissions.

Enfin, mes travaux fournissent les outils et une méthode déterminer les meilleurs stratégies au vu des incertitudes actuelles. Ce travail pourrait être mis à jour régulièrement afin d’intégrer les nouvelles informations en termes de coûts, de demande d’électricité ou de prix du CO\textsubscript{2}. 

Du côté de l’emploi, j’aurais voulu poursuivre mon analyse en m’intéressant aux effets transitoires et au rôle de la monnaie, par exemple en mobilisant des modèles de type nouveau-keynésien, comme dans l'article de \citet{Blanchard2007} observant l'impact d'un choc pétrolier\footnote{«~The Macroeconomic Effects of Oil Price Shocks: Why are the 2000s so different from the 1970s?~»}.
Plus fondamentalement, mes questionnements ont porté sur la pertinence et l’importance des choix de modélisations sur les résultats. Cela s’est traduit par la comparaison de modèles dans le chapitre \ref{chap:mechanisms}, et l’analyse de nombreuses spécifications différentes et tests de sensibilité : sur l’ajustement de la balance commerciale, le marché de travail, etc. 
Mais ces interrogations sur les modèles pourraient être étendues à d'autres hypothèses plus fondamentales encore. En particulier le questionnement de l’utilisation d’un agent représentatif, parfaitement rationnel et informé, me paraît un axe de recherche attractif. Bien sûr, les critiques de cet \textit{homo economicus} ne sont pas nouvelles. Carl Menger critiquait l’hypothèse implicite d’homogénéité des individus ; Keynes l’hypothèse d’information complète ; Veblen la rationalité parfaite ; \citet{Bourdieu1977} l’a qualifié de «~monstre anthropologique~», et \citet{Sen2012} de «~demeuré social~» ; plus récemment, les travaux de l’économie comportementale ont mis en évidence de nombreux biais cognitifs \citep{Thaler2009,Kahneman2011}.
Mais l’ampleur des critiques souligne le chemin qui reste à parcourir. Des pistes qui me semblent intéressantes ont trait à la compréhension et la modélisation du comportement des agents. L'hypothèse de «~rationalité parfaite~», malgré son appellation flatteuse, est finalement une représentation très stylisée de la façon dont nous prenons une décision. Il devrait donc être possible d'enrichir cette représentation, de se rapprocher des comportements observés dans les autres sciences sociales. Par exemple, des travaux récents explorent l’impact d’agents «~presque parfaitement rationnels~» (near-rational expectations). Ainsi, \citet{Farhi2016} montre que l’effet des politiques monétaires est alors plus faible que ce qu’indiquent les modèles à agents rationnels.

Par ailleurs, l’explosion des capacités computationnelles ouvre le champ à l’introduction de modèles avec de nombreux agents représentatifs – plusieurs centaines ou milliers –, ce qui pourrait permettre de dégager des dynamiques agrégées qui échappent à des modèles représentant seulement un ou quelques agents représentatifs. Des travaux récents vont dans cette direction. Ainsi, \citet{Kaplan2016} montrent que les effets de la politique monétaires sont très différents dans des modèles à agents représentatifs et dans des modèles à agents hétérogènes. 

Représenter explicitement des agents multiples, dans leur diversité et leur imperfection, me paraît être une piste de recherche intéressante à double titre. Elle peut permettre à l’économie de renforcer ses fondements empiriques, de poser ainsi des hypothèses vérifiables et réfutables – critère de scientificité selon Popper – et d’abandonner, le cas échéant, des hypothèses contredites par l’expérience. Englober la théorie du choix rationnel dans un modèle plus global de l’action peut être l’occasion de renouer des liens avec les autres sciences sociales, notamment l’anthropologie, la psychologie et la sociologie. «~L'économie qui est la science sociale mathématiquement la plus avancée, est la science socialement la plus arriérée, car elle s'est abstraite des conditions sociales, historiques, politiques, psychologique, écologiques inséparables des activités économiques~» constate \citet{Morin1999}. Sortir de ce fonctionnement en silos, se rapprocher des autres disciplines pour mieux appréhender la complexité de l'humain : voilà qui me paraît une piste prometteuse de recherche future.





