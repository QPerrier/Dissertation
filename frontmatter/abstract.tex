%Introduction

This thesis deals with two aspects of the transition towards a low-carbon economy: how to define a strategy under the uncertainties and inertia surrounding power systems, and what are the impacts of this transition on employment?

%%%%%%%%%%%%%
%Stratégie robuste
To study the choice of a strategy, I focus on the case of the French power system. The 58 nuclear reactors are reaching the end of their initially planned lifetime, but they can be retrofitted for a total cost estimated at a \euro 100 billion. How many reactors should be retrofitted? I study this question using an optimization model of the French power system and the \textit{Robust Decision Making} framework. My results indicate that the most interesting strategies consist in closing 7 to 14 reactors in favor of renewable energies, given current estimates.

%%%%%%%%%%%%%%
%Contenu en emploi
As to employment impacts, I study the notion of employment content and offer an original methodology to break it down and understand the relative importance of four components: import rates, taxes and subsidies, the share of labour in value added and the level of wage.
Then, I study the employment impacts of shifting investment towards low-carbon sectors with CGE and Input-Output models, and highlight the underlying economic mechanisms.
My results suggest that encouraging sectors with a high share of labor or low wages might increase employment, but they also challenge the alleged benefits of targeting sectors with a low import rate.