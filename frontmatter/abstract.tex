%Introduction

This thesis deals with two aspects of the transition towards a low-carbon economy: how to define a strategy under the uncertainties and inertia surrounding power systems, and what are the impacts of this transition on employment?

%%%%%%%%%%%%%
%Stratégie robuste
To study the choice of a strategy, I focus on the case of the French power system. Nuclear plants are reaching the end of their initially planned lifetime, but they can be retrofitted with an investment. How many reactors should be retrofitted? I study this question using the \textit{Robust Decision Making} framework. I show that several strategies can be considered given the range of uncertainties, but given current estimates, the most interesting ones consist in closing 12 to 17 reactors in favor of renewable energies.

%%%%%%%%%%%%%%
%Contenu en emploi
On another side, I study the notion of employment content and offer an original methodology to break it down and understand the relative importance of four components: import rates, taxes and subsidies, the share of labour in value added and the level of wage.
Then, I study the employment impact of shifting investment towards low-carbon sectors, and highlight its underlying economic mechanisms.
My results suggest that encouraging sectors with a high share of labor or low wages might increase employment, but they also challenge the idea that targeting sectors with low import rate creates jobs.