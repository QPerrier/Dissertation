%Introduction
Dans cette thèse, je discute deux questions autour de la transition énergétique : comment définir une stratégie face aux nombreuses incertitudes et aux inerties des systèmes, et quels sont impacts sur l’emploi de cette transition ?

%%%%%%%%%%%%%
%Stratégie robuste
Pour étudier le choix d’une stratégie, je m’intéresse au cas du secteur électrique français. Les réacteurs nucléaires arrivent au terme de leur durée de vie initiale, mais ils peuvent être rénovés moyennant un investissement estimé à 100 milliards d'euros pour l'ensemble du parc. Combien de centrales faut-il rénover ? En mobilisant un modèle d'optimisation et la méthode de \textit{Robust Decision Making}, je montre qu’au vu des estimations actuelles, les stratégies les plus intéressantes consistent à fermer 7 à 14 réacteurs et à les remplacer par des énergies renouvelables.

%%%%%%%%%%%%%%
%Contenu en emploi
Sur le volet de l’emploi, je m’intéresse tout d’abord à la notion de contenu en emploi. Je propose une méthodologie nouvelle permettant de décomposer ce contenu, afin de mettre en évidence, pour chaque branche économique, l’importance relative de ses constituants : les taux d’importations, les taxes et subventions, la part du travail dans la valeur ajoutée et le niveau de salaire. 

J'étudie ensuite l’impact d’une réallocation des investissements vers des secteurs bas-carbone à l'aide de modèles d'équilibre général, dont j'explicite les mécanismes économiques sous-jacents.
Mes résultats indiquent qu'encourager les secteurs avec une forte part du travail dans la valeur ajoutée ou de faibles salaires permet de créer de l'emploi, mais ils nuancent les bénéfices à encourager des secteurs peu importateurs.
