%Introduction
Cette thèse discute deux questions autour de la transition énergétique : comment définir une stratégie face aux nombreuses incertitudes et aux inerties des systèmes ? quels sont impacts sur l’emploi de cette transition ?

%%%%%%%%%%%%%
%Stratégie robuste
Pour étudier le choix d’une stratégie, je m’intéresse au cas du secteur électrique français. Les centrales nucléaires arrivent à l’échéance de leur durée de vie initialement prévue, mais elles peuvent être rénovées moyennant un investissement dit de « Grand Carénage ». Combien de centrales faut-il rénover ? J’étudie cette question en appliquant la méthodologie de \textit{Robust Decision Making}. Je montre qu’au vu des estimations actuelles, les stratégies les plus intéressantes consistent à fermer 7 à 14 réacteurs et à les remplacer par des énergies renouvelables.

%%%%%%%%%%%%%%
%Contenu en emploi
Sur le volet de l’emploi, je m’intéresse tout d’abord à la notion de contenu en emploi. Je propose une méthodologie nouvelle permettant de décomposer ce contenu, afin de mettre en évidence, pour chaque branche économique, l’importance relative de ses constituants : les taux d’importations, les taxes et subventions, la part du travail dans la valeur ajoutée et le niveau de salaire. 
%Nous évaluons ensuite le contenu en emploi et en émissions de gaz à effet de serre de toutes les branches économiques françaises en 2010, pour étudier les substitutions interbranches d’une transition énergétique. Nos résultats indiquent que les variations de contenu en emploi entre branches s’expliquent, dans l’ordre, par le niveau de salaire, la part du travail dans la valeur ajoutée, le taux d’importations finales, le taux d’importations intermédiaires, et en dernier par les taxes et subventions. Par ailleurs, nous montrons que l’EU ETS couvre les branches intensives en émissions et peu intensives en em- ploi, mais pas les branches intensives en émissions et en emploi. L’emploi pourrait donc expliquer en partie le choix des branches soumises à l’EU ETS. Enfin, nous identifions des substitutions qui favoriseraient des branches moins intensives en émissions et plus intensives en emploi.
J'étudie ensuite l’impact d’une reallocation des investissements vers des secteurs bas-carbone, afin d'expliciter les mécanismes économiques sous-jacents aux impacts sur l'emploi. 
Je montre qu'encourager les secteurs avec une forte part du travail dans la valeur ajoutée ou de faibles salaires permet de créer de l'emploi. Mes résultats nuancent en revanche les bénéfices présumés à encourager des secteurs peu importateurs.


%Using stylized CGE and IO models, we identify and discuss three channels of job creation due to an investment shift: positive employment impacts arise from targeting sectors with a high labour share in value added, low wages or low import rates. Results are robust across both models, except for the latter that only occurs in IO. We then undertake a numerical analysis of two policies: solar panel installation and weatherproofing. These investments both yield a positive effect on employment, a result which is robust across models, due to a high share of labour and low wages in these sectors. The results are roughly similar in IO and CGE for solar; for weatherproofing, the results are higher in IO because of low import rates, by a factor ranging from 1.19 to 1.87. Our conclusions challenge the idea that renewable energies boost employment by reducing imports, but they also suggest that an employment double dividend might exist when encouraging low-carbon labour-intensive sectors.
