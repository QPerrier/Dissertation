% the acknowledgments section

Trois ans, pour le meilleur et pour le pire. Trois ans à trouver des questions et chercher des réponses, à s’enthousiasmer et à douter, travailleur de la preuve, interprète du chiffre. 

Dans ces pérégrinations intellectuelles et émotionnelles, ma reconnaissance va d’abord à mon directeur de thèse, Philippe. Mille mercis pour ta disponibilité, ton soutien et tes conseils. Je m'efforcerai de conserver toujours cette rigueur scientifique que tu m'as inculquée. 

A Daniel, merci de m'avoir fait confiance en me recrutant, ainsi que pour ton soutien et la liberté de recherche qui m'ont permis d'effectuer ma thèse dans les meilleures conditions.

Je remercie mes rapporteurs, pour leur relecture soigneuse de mon manuscrit : Patrick Criqui et Jean de Beir, ainsi que les autres membres du jury : Michel Colombier et Jean-Charles Hourcade et Katheline Schubert. Merci également à Franck Lecocq pour m'avoir accueilli au laboratoire du Cired.

Je tiens à exprimer ma gratitude à tous les collègues qui ont pris le temps de m’aider dans mes recherches, notamment Jean sur les modèles CGE, Manuel sur les modèles du parc électrique et Céline sur les incertitudes radicales.

Le jardin tropical est un paradis caché par sa flore foisonnante ; et en son sein, le Cired l’est doublement par sa faune bigarrée. Mes pensées vont d’abord aux tauliers de l’open space : Rémi le slackeur, Jules l'engagé, Basile, Nicolas, Vivien, Quentin, Mathilde et Caroline. Je remercie également tous les autres, collègues et amis, qui participent à cet esprit du Cired : Vincent, Audrey, Florian, William, Gaëlle, Aurélien, Jules, Yann, le \textit{master of code} Patrice, les pongistes Laurent, Thierry et Am\textit{é}line, les batteurs de cloche Antoine et Aurélie, Eléonore, Naceur, Estelle et ses gâteaux pleins de bonne humeur, Maha, Laurent, Cyril, Elsa, Anne, notre statisticien illuminati qui se reconnaitra, Christophe, Harold, Gaëtan, Julien, Alain, Meriem et tous les autres. 
Et je n’oublie pas les jeunes «~anciens~» qui nous ont quittés : Noémie, Béatrice, Manu, Marie-Laure, Eoin, Simona…

Je garde également de précieux souvenirs des moments à Engie. Le bureau des trois mousquetaires, Paul, Félix et Valentin, avec leurs discussions passionnantes voire passionnées. Le quartier général de l'information géré avec brio par les deux Elodies. Les pauses avec Ibrahim aka Maître GAMS, Antoine, Florian et Medhi. Les échanges sur le pas de la porte avec Elie, les trajets avec Aurélien, et tous les autres : Guillaume, Laëtitia, Emilie, Isabelle, Joël, Adrien... Avec, bien sûr, une mention spéciale à Chimène, notre incroyable professeure de piano.

Je pense aussi à tous mes amis passionnés de politique et d’environnement : Mathieu, Pauline, Caroline, JB, Cécile, Philippe – même si vous prenez l’avion tous les quinze jours ; Pierre et Clémentine, pour votre courage dans votre engagement ; Lucile, mon agent infiltré au RAC et Alexis qui promeut la paix verte.

Merci au gang des «~grosses~» pour toutes nos parties de Seven Wonders, nos soirées GOT et nos brunchs. A la bande des mineurs, pour nos eskapades par monts et par flots. Aux copains de lycée avec qui j’apprécie toujours autant d’arpenter les troquets de Saint-Martin. A Cédric qui m’a lancé sur les pistes de danse. Au PFC \& Co pour m’avoir fait redécouvrir la chanson française, les vadrouilles en vélo et les week-ends à Labrosse. Aux zikos de mon c\oe{}ur, Vincent, Valère et FX, pour nos séances de jazz du jeudi soir en mode pinte atomique.

Enfin, mes pensées vont à ma famille, qui m’a encouragé dans cette thèse et s’est même passionnée pour ce sujet : mes parents, Claire et Franck, ma philosophe de sœur Marine, mes grands-parents Jean et Gabrielle. A Thomas, Dominique et Garance, pour m’avoir fait découvrir l’Anthropocène, et à Paul pour ses suggestions d’expos et de restaurants. 

Je conclus évidemment avec Essinev, ma première lectrice, ma conseillère et ma source d’énergie, et, sans aucun doute, l’être social le plus doué que j’aie jamais rencontré. Merci pour tous ces moments partagés, passés et à venir. 

